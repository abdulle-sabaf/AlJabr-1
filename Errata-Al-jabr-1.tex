%!TEX TS-program = xelatex
%!TEX encoding = UTF-8

% LaTeX source for the errata of the book ``代数学方法'' in Chinese
% Copyright 2023  李文威 (Wen-Wei Li).
% Permission is granted to copy, distribute and/or modify this
% document under the terms of the Creative Commons
% Attribution 4.0 International (CC BY 4.0)
% http://creativecommons.org/licenses/by/4.0/

% 《代数学方法》卷一勘误表 / 李文威
% 使用自定义的文档类 AJerrata.cls. 自动载入 xeCJK.

\documentclass{AJerrata}

\usepackage{unicode-math}

\usepackage[unicode, colorlinks, psdextra, bookmarksnumbered,
	pdfpagelabels=true,
	pdfauthor={李文威 (Wen-Wei Li)},
	pdftitle={代数学方法卷一勘误},
	pdfkeywords={}
]{hyperref}

\setmainfont[
	BoldFont={texgyretermes-bold.otf},
	ItalicFont={texgyretermes-italic.otf},
	BoldItalicFont={texgyretermes-bolditalic.otf},
	PunctuationSpace=2
]{texgyretermes-regular.otf}

\setsansfont[
	BoldFont=FiraSans-Bold.otf,
	ItalicFont=FiraSans-Italic.otf
]{FiraSans-Regular.otf}

\setCJKmainfont[
	BoldFont=Noto Serif CJK SC Bold
]{Noto Serif CJK SC}

\setCJKsansfont[
	BoldFont=Noto Sans CJK SC Bold
]{Noto Sans CJK SC}

\setCJKfamilyfont{emfont}[
	BoldFont=FandolHei-Regular.otf
]{FandolHei-Regular.otf}	% 强调用的字体

\renewcommand{\em}{\bfseries\CJKfamily{emfont}} % 强调

\setmathfont[
	Extension = .otf,
	math-style= TeX,
]{texgyretermes-math}

\usepackage{mathrsfs}
\usepackage{stmaryrd} \SetSymbolFont{stmry}{bold}{U}{stmry}{m}{n}	% 避免警告 (stmryd 不含粗体故)
% \usepackage{array}
% \usepackage{tikz-cd}  % 使用 TikZ 绘图
\usetikzlibrary{positioning, patterns, calc, matrix, shapes.arrows, shapes.symbols}

\usepackage{myarrows}				% 使用自定义的可伸缩箭头
\usepackage{mycommand}				% 引入自定义的惯用的命令


\title{\bfseries 代数学方法(第一卷)勘误表 \\ 跨度: 2023 迄今 }
\author{李文威}
\date{\today}

\begin{document}
	\maketitle
	以下页码涉及代数学方法(第一卷)修订版.

	\begin{Errata}
		\item[例 2.1.5 第 1 项第一行]
		\Orig 任两个对象间至多只有一个态射的范畴
		\Corr 对任一对对象 $(X, Y)$ 至多只有一个态射 $X \to Y$ 的范畴
		\Thx{感谢彭行一指正}
		
		\item[例 2.1.5 第 7 项]
		\Orig $\mathrm{Vect}_f$
		\Corr $\mathsf{Vect}_f$
		%\Thx{}
		
		\item[定义 2.3.1 第二项 (余积)]
		将所有 $X_k$ 改成 $X'_k$ (两处). 另外将最后一行的 $X_j \in \Obj(\mathcal{C}_j)$ 改成 $X_j, X'_j \in \Obj(\mathcal{C}_j)$.
		\Thx{感谢 Alissa Tung 指正}
		
		\item[命题 2.6.9 证明第二行]
		\Orig $h_{\mathcal{C}}(GY)$
		\Corr $h_{\mathcal{C}_1}(GY)$
		\Thx{感谢雷嘉乐指正}
		
		\item[定理 2.6.12 证明]
		\Orig 等式右边的底部再装配 $\epsilon$...
		\Corr 等式右边的底部再装配 $\varepsilon$...
		\Thx{感谢雷嘉乐指正}
		
		\item[定义 4.3.7 陈述的最后一则公式]
		\Orig $\Image(G)$
		\Corr $\Image(\varphi)$
		\Thx{感谢李隆平指正}
		
		\item[第二章习题 10]
		\Orig $\cate{Vect}_f(\Bbbk)$
		\Corr $\cate{Vect}(\Bbbk)$
		\Thx{感谢雷嘉乐指正}
		
		\item[例 3.3.8, 第 85 页 Artin 辫群的定义之上]
		\Orig 两条垂直线 $\vert\;\vert$
		\Corr 三条垂直线 $\vert\;\vert\;\vert$
		\Thx{感谢刘欧指正}
		
		\item[定理 5.8.7 的陈述]
		\Orig $(-1)^k k e_k$
		\Corr $k e_k$
		\Thx{感谢雷嘉乐指正}
		
		\item[公式 (7.7) 之下第三行]
		\Orig $A_i \otimes B_j$
		\Corr $A_j \otimes B_k$
		\Thx{感谢雷嘉乐指正}
		
		\item[公式 (7.12) 之上第二行]
		\Orig $\cdots < i_l \leq n$
		\Corr $\cdots < i_k \leq n$
		\Thx{感谢雷嘉乐指正}
		
		\item[定义 7.8.3 之上第三行]
		\Orig $s \cdot \Tr(\varphi)$
		\Corr $s \cdot \Tr(\psi)$
		\Thx{感谢雷嘉乐指正}
		
		\item[定义 9.3.3 之下第二个交换图表右上角]
		\Orig $\varphi(b)$
		\Corr $\varphi(a)$
		\Thx{感谢雷嘉乐指正}
		
		\item[命题 9.4.2 陈述]
		\Orig 而且 $\mu_n$ 是...
		\Corr 而且 $\mu_n(\overline{F})$ 是...
		\Thx{感谢雷嘉乐指正}
		
		\item[定理 9.4.6 证明第一句]
		\Orig $\Q(\mu_n)$
		\Corr $\Q(\zeta_n)$
		\Thx{感谢雷嘉乐指正}
		
		\item[命题 10.3.5 陈述第二行]
		\Orig $v(\varpi)^k$
		\Corr $v(\varpi^k)$
	\end{Errata}
\end{document}
