%!TEX TS-program = xelatex
%!TEX encoding = UTF-8

% LaTeX source for the errata of the book ``代数学方法'' in Chinese
% Copyright 2025  李文威 (Wen-Wei Li).
% Permission is granted to copy, distribute and/or modify this
% document under the terms of the Creative Commons
% Attribution 4.0 International (CC BY 4.0)
% http://creativecommons.org/licenses/by/4.0/

% 《代数学方法》卷一勘误表 / 李文威
% 使用自定义的文档类 AJerrata.cls. 自动载入 xeCJK.

\documentclass{AJerrata}

\usepackage{unicode-math}

\usepackage[unicode, colorlinks, psdextra, bookmarksnumbered,
	pdfpagelabels=true,
	pdfauthor={李文威 (Wen-Wei Li)},
	pdftitle={代数学方法卷一勘误},
	pdfkeywords={}
]{hyperref}

\setmainfont[
	BoldFont={texgyretermes-bold.otf},
	ItalicFont={texgyretermes-italic.otf},
	BoldItalicFont={texgyretermes-bolditalic.otf},
	PunctuationSpace=2
]{texgyretermes-regular.otf}

\setsansfont[
	BoldFont=FiraSans-Bold.otf,
	ItalicFont=FiraSans-Italic.otf
]{FiraSans-Regular.otf}

\setCJKmainfont[
	BoldFont=Noto Serif CJK SC Bold
]{Noto Serif CJK SC}

\setCJKsansfont[
	BoldFont=Noto Sans CJK SC Bold
]{Noto Sans CJK SC}

\setCJKfamilyfont{emfont}[
	BoldFont=FandolHei-Regular.otf
]{FandolHei-Regular.otf}	% 强调用的字体

\renewcommand{\em}{\bfseries\CJKfamily{emfont}} % 强调

\setmathfont[
	Extension = .otf,
	math-style= TeX,
]{texgyretermes-math}

\usepackage{mathrsfs}
\usepackage{stmaryrd} \SetSymbolFont{stmry}{bold}{U}{stmry}{m}{n}	% 避免警告 (stmryd 不含粗体故)
% \usepackage{array}
% \usepackage{tikz-cd}  % 使用 TikZ 绘图
\usetikzlibrary{positioning, patterns, calc, matrix, shapes.arrows, shapes.symbols}

\usepackage{myarrows}				% 使用自定义的可伸缩箭头
\usepackage{mycommand}				% 引入自定义的惯用的命令


\title{\bfseries 代数学方法(第一卷)勘误表 \\ 跨度: 2023 --- 2024 }
\author{李文威}
\date{\today}

\begin{document}
	\maketitle
	以下页码等信息参照高等教育出版社 2023 年 2 月重印之《代数学方法》第一卷, ISBN: 978-7-04-050725-6. 这些错误将在下一批重印的版本改正.

	\begin{Errata}
		\item[定理 3.4.9 证明第一段结尾处]
		\Orig 唯一确定了 $\varphi$. 因此...
		\Corr 唯一确定了 $\phi$. 因此...
		\Thx{感谢刘欧指正}
		
		\item[例 2.1.5 第 1 项第一行]
		\Orig 任两个对象间至多只有一个态射的范畴
		\Corr 对任一对对象 $(X, Y)$ 至多只有一个态射 $X \to Y$ 的范畴
		\Thx{感谢彭行一指正}
		
		\item[例 2.1.5 第 7 项]
		\Orig $\mathrm{Vect}_f$
		\Corr $\mathsf{Vect}_f$
		%\Thx{}
		
		\item[例 2.2.9]
		将显示公式第一行的 $\cate{CHaus}$ 换成 $\cate{CHaus}^{\text{op}}$
		\Thx{感谢毕家烨指正}
		
		\item[定义 2.3.1 第二项 (余积)]
		将所有 $X_k$ 改成 $X'_k$ (两处). 另外将最后一行的 $X_j \in \Obj(\mathcal{C}_j)$ 改成 $X_j, X'_j \in \Obj(\mathcal{C}_j)$.
		\Thx{感谢 Alissa Tung 指正}
		
		\item[命题 2.6.9 证明第二行]
		\Orig $h_{\mathcal{C}}(GY)$
		\Corr $h_{\mathcal{C}_1}(GY)$
		\Thx{感谢雷嘉乐指正}
		
		\item[定理 2.6.12 证明]
		\Orig 等式右边的底部再装配 $\epsilon$...
		\Corr 等式右边的底部再装配 $\varepsilon$...
		\Thx{感谢雷嘉乐指正}
		
		\item[定义 2.7.2 之下的讨论]
		\Orig 余锥和锥
		\Corr 锥和余锥
		\Thx{感谢黄行知指正}
		
		\item[\S 2.7, 公式 (2.11) 之后的图表]
		右图从 $x_j$ 出发的两个箭头从 $\to$ 改成 $\mapsto$.
		\Thx{感谢陈思成指正}
		
		\item[第二章习题 10]
		\Orig $\cate{Vect}_f(\Bbbk)$
		\Corr $\cate{Vect}(\Bbbk)$
		\Thx{感谢雷嘉乐指正}
		
		\item[定义 3.1.7 的交换图表右上角的项]
		\Orig $Y \times Z$
		\Corr $Y \otimes Z$
		
		\item[例 3.3.8, 第 85 页 Artin 辫群的定义之上]
		\Orig 两条垂直线 $\vert\;\vert$
		\Corr 三条垂直线 $\vert\;\vert\;\vert$
		\Thx{感谢刘欧指正}
		
		\item[第三章习题 1]
		\Orig ... $X_1, \ldots, X_n$ ... 它们的 $n$-重 ...
		\Corr ... $X_1, \ldots, X_{n+1}$ ... 循序的 $n$-重 ...
		\Thx{感谢李隆平指正}
		
		\item[定义 4.3.7 陈述的最后一则公式]
		\Orig $\Image(G)$
		\Corr $\Image(\varphi)$
		\Thx{感谢李隆平指正}
		
		\item[定义 4.8.1 第三行]
		\Orig $\varphi: \mathbf{M}(X) \to M$
		\Corr $\varphi: \mathbf{M}(X) \to M'$
		\Thx{感谢王继麟指正}
		
		\item[(4.6) 以下的讨论]
		\Orig 在 $M_1$ 中可写...
		\Corr 在 $M_{i_1}$ 中可写...
		\Thx{感谢曲锐恒指正}
		
		\item[引理 4.9.5 证明第三行]
		\Orig $\sigma(f \pm g) = \sigma f \pm \sigma g$, 其中 $k \in \Z$.
		\Corr $\sigma(f \pm g) = \sigma f \pm \sigma g$.
		\Thx{感谢蓝青指正}
		
		\item[第四章习题 26]
		将 $\varinjlim_U \varprojlim_V$ 换成 $\varprojlim_V \varinjlim_U$.
		
		\item[引理 5.4.5 证明最后的公式]
		\Orig $\displaystyle\sum_{x_1 \leq z_1 \leq y_n}$
		\Corr $\displaystyle\sum_{x_1 \leq z_1 \leq y_1}$
		
		\item[例 5.4.7 第三行]
		删除 ``(即保序双射)''
		
		\item[例 5.4.7 第二个显示公式的第一项]
		\Orig $\mu\left( \prod_p n_p, \prod_p m_p \right)$
		\Corr $\mu\left( \prod_p p^{n_p}, \prod_p p^{m_p} \right)$
		
		\item[命题 5.6.5 的陈述中部]
		\Orig 若 $f$ 和 $g$ 的像在 $S$ 中对乘法相交换, ...
		\Corr 若 $f$ 和 $g$ 的像在 $S$ 中对乘法相交换, $f$ 的像对乘法也交换, ...
		\Thx{感谢褚浩云指正}
		
		\item[定理 5.7.9 证明中第一个列表的第二项]
		\Orig $\bar{\mathfrak{p}} = \mathfrak{p}$
		\Corr $\mathring{\mathfrak{p}} = \mathring{\bar{\mathfrak{p}}}$
		\Thx{感谢王继麟指正}
		
		\item[定理 5.8.7 的陈述]
		\Orig $(-1)^k k e_k$
		\Corr $k e_k$
		\Thx{感谢雷嘉乐指正}
		
		\item[第五章习题 10]
		\Orig $Z(P, n) := \zeta^n(\hat{0}, \hat{1})$
		\Corr $Z(P, n)$ 为 $P$ 中的列 $x_1 \leq \cdots \leq x_{n-1}$ 的个数.
		\Thx{感谢毕家烨指正}
		
		\item[注记 6.2.3 的显示公式]
		应将 $\bigoplus$ 改成 $\bigsqcup$, 下标不变.

		\item[引理 6.3.5 陈述最后一行] 在 $\Hom(\cdots) \times \Hom(\cdots)$ 中对调两个 $\Hom$ 的位置.

		\item[例 6.5.2 之上的最后一句]
		\Orig ...化到单模的情形.
		\Corr ...化到单边的情形.

		\item[命题 6.5.11]
		命题陈述中两行公式之间的左侧 $\rotatebox{90}{$\subset$}$ 改成箭头 $\rotatebox{90}{\to}$. 另外, 证明第五行的``两个同态''改为``两个横向同态''.
		\Thx{感谢毕家烨指正}
		
		\item[命题 6.5.13 证明第三行中间]
		\Orig $M \dotimes{S} S \xrightarrow{\text{平衡积}} \cdots$
		\Corr $M \times S \xrightarrow{\text{平衡积}} \cdots$
		
		\item[定理 6.9.10 证明倒数第四行]
		\Orig $\stackrel{r \mapsto rx}{\hookrightarrow}$.
		\Corr $\xrightarrow{r \mapsto rx}$.
		\Thx{感谢蓝青指正}

		\item[定理 6.10.7 证明]
		证明结尾处延续原来段落, 补上以下文字: ``最后一步改为用形如 $\sum_{i=1}^m u_i f_i X^{d_i}$ 的元素不断消去 $f_{m+1}$ 的最低次项, 最终推得 $f_{m+1} \in \lrangle{f_1, \ldots, f_m}$.
		\Thx{感谢毕家烨指正}
		
		\item[引理 6.11.3 之上第二第三行]
		\Orig $\End_R(M) \rightiso \cdots$
		\Corr
		\[ \End_R(M)^{\text{op}} \rightiso \prod_{i=1}^n M_{n_i}\left(D_i^{\text{op}}\right)^{\text{op}} \xrightarrow[\text{转置}]{\sim} \prod_{i=1}^n M_{n_i}(D_i) \]
		
		\Orig $M_i^{\oplus n_i}$ 是右 $D_i$-模
		\Corr $M_i^{\oplus n_i}$ 是右 $M_{n_i}(D_i)$-模
		\Thx{感谢蓝青指正}

		\item[第六章习题 10]
		\Orig $\iff b \in \Q\pi$
		\Corr $\iff a = 0 \;\text{或}\; b \in \Q\pi$
		\Thx{感谢王继麟指正}

		\item[7.1 节倒数第二段的公式之前]
		\Orig $M_n$ 是自由左 $A$-模:
		\Corr $M_n(A)$ 是自由左 $A$-模:
		\Thx{感谢李隆平指正}

		\item[公式 (7.7) 之下第三行]
		\Orig $A_i \otimes B_j$
		\Corr $A_j \otimes B_k$
		\Thx{感谢雷嘉乐指正}
		
		\item[引理 7.6.4 证明中部]
		\Orig $(\sigma B)(x_1, \ldots, x_n) = B(x_{\sigma^{-1}(1)}, \ldots, x_{\sigma^{-1}(n)})$
		\Corr $(\sigma B)(x_1, \ldots, x_n) = B(x_{\sigma(1)}, \ldots, x_{\sigma(n)})$
		\Thx{感谢蓝青指正}
		
		\item[推论 7.6.9 证明之下第六行]
		\Orig $T^n_\chi := \cdots$
		\Corr $T^n_\chi(M) := \cdots$
		\Thx{感谢蓝青指正}
		
		\item[公式 (7.12) 之上第二行]
		\Orig $\cdots < i_l \leq n$
		\Corr $\cdots < i_k \leq n$
		\Thx{感谢雷嘉乐指正}
		
		\item[定义 7.8.3 之上第三行]
		\Orig $s \cdot \Tr(\varphi)$
		\Corr $s \cdot \Tr(\psi)$
		\Thx{感谢雷嘉乐指正}
		
		\item[定理 7.8.5 陈述]
		第二个等式的 $\mathrm{N}_R(\varphi)$ 改为 $\det_R(\varphi)$.
		\Thx{感谢毕家烨指正}
		
		\item[第七章习题 6 (iii)]
		将显示公式第二行的 ``$A$ 交换'' 改为 ``$A$ 结合交换''
		\Thx{感谢毕家烨指正}
		
		\item[定义--定理 8.3.4 证明]
		倒数第一和第二行的两处 $R_x$ 应改为 $R_P$.
		\Thx{感谢李隆平指正}
		
		\item[定义 9.3.3 之下第二个交换图表右上角]
		\Orig $\varphi(b)$
		\Corr $\varphi(a)$
		\Thx{感谢雷嘉乐指正}
		
		\item[定理 9.3.4 证明第二行]
		\Orig $\Gal(E|F) = \cdots$
		\Corr $|\Gal(E|F)| = \cdots$
		\Thx{感谢蓝青指正}
		
		\item[命题 9.4.2 陈述]
		\Orig 而且 $\mu_n$ 是...
		\Corr 而且 $\mu_n(\overline{F})$ 是...
		\Thx{感谢雷嘉乐指正}
		
		\item[定理 9.4.6 证明第一句]
		\Orig $\Q(\mu_n)$
		\Corr $\Q(\zeta_n)$
		\Thx{感谢雷嘉乐指正}
		
		\item[公式 (9.11), 及其下两处]
		将 $\chi(\Delta, \gamma) \xlongequal{\text{恒等}} 1$, $\chi(a, \Gamma) \xlongequal{\text{恒等}} 1$, $\chi(\Delta, \gamma) = 1$ 和 $\chi(a, \Gamma_E) = 1$ 中的 $1$ 全部改为 $0$.
		\Thx{感谢毕家烨指正}
		
		\item[第九章习题 13]
		在``无关根的排序.'' 之后加一句``设 $\mathrm{char}(F) \neq 2$''.
		\Thx{感谢毕家烨指正}
		
		\item[第九章习题 17]
		\Orig ...可约则 $G \simeq D_8$...
		\Corr ...不可约则 $G \simeq D_8$...
		\Thx{感谢毕家烨指正}
		
		\item[例 10.1.3 列表第二项结尾]
		\Orig $\cdots \implies E \in \mathfrak{N}_y$
		\Corr $\cdots \implies F \in \mathfrak{N}_y$
		\Thx{感谢黄行知指正}
		
		\item[例 10.1.3 最后一段]
		引用文献的定理 2.2.3 改为定理 2.3.3.
		
		\item[命题 10.3.5 陈述第二行]
		\Orig $v(\varpi)^k$
		\Corr $v(\varpi^k)$
		
		\item[第十章习题 18]
		\Orig 推论 10.6.8
		\Corr 推论 10.7.8
		\Thx{感谢毕家烨指正}
	\end{Errata}
\end{document}
